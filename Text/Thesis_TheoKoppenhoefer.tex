

\input{../Latex_Templates/Preamble_Report}

%%%%% TITLE PAGE

%\subject{, VT23}
\title{ Some title \\[1ex]
	  \large Master Thesis}
%\subtitle{}
\author{Theo Koppenhöfer}
\date{Lund \\[1ex] \today}

\addbibresource{bibliography.bib}

\usepackage{pythonhighlight}
\usepackage{pgfplots}
\graphicspath{{../Plots/}}

\pgfplotsset{
	compat=newest,
    every axis/.append style={
        axis y line=left,
        axis x line=bottom,
        scale only axis,
%    	max space between ticks=25pt,
        width=0.7\textwidth,
        scaled ticks=true,
        axis line style={thick,-,>=latex, shorten >=-.4cm},
    		x tick label style={
		    /pgf/number format/precision=3
		}
    },
    every axis plot/.append style={thick},
    tick style={black, thick}
}

\newcommand{\bx}{\bar{x}}

%%%%% The content starts here %%%%%%%%%%%%%


\begin{document}

\maketitle

Some amazing introduction
 
\newpage

\section*{Some general remarks}

\subsection*{On assuming non-degeneracy}

\newpage

\section*{The case $n=2$}

\begin{claim}
  Let $\Omega$ be homoemorph to $B_1\subseteq\R^2$. Let further $f\colon\overline{\Omega}\to\R$ be harmonic and admissable 
  as in Morse with critical point $x_0\in\Omega$. Then $\Sigma^-\subseteq\partial\Omega$ consists fo at least 2 components.
\end{claim}


\subsection*{A proof involving level-sets}

\begin{proof}
  Let $y_c=f(x_0)$ and $x_0,\dots,x_N$ be all the critical points s.t.\ $f(x_i)=y_c$. We claim that the set
  \begin{align*}
    C=\brk[c]*{f=y_c}\subseteq\overline{\Omega}
  \end{align*}
  divides $\partial\Omega$ into 4 components. To show this let $\gamma_i\colon \brk*{a_i,b_i}\to C$ parametrise the curves in
  $C$ intersecting at $x_0$. These can be constructed with the initial value problem
  \begin{align*} 
    \gamma'=\brk*{Df}^\perp\big\vert_\gamma \\
    \gamma(0) = \gamma_0
  \end{align*}
  where $\gamma_0\in C$ is chosen sufficiently near $x_0$. Without loss of generality the intervals on which the $\gamma_i$ are
  defined are maximal. We thus have for
  \begin{align*}
    \gamma_i^-=\lim_{t\to a_i}\gamma(t) \\
    \gamma_i^+=\lim_{t\to b_i}\gamma(t)
  \end{align*}
  that $\gamma_i^\pm\in\brk*[c]{x_0,\dots,x_N,\partial\Omega}$ since the $x_j$ are the sole points on $\Omega\cap\overline{C}$
  at which $Df^\perp=0$. We therefore have a situation similar to the one depicted in [TODO: make figure].
  One sees that $C$ can thus be represented by a graph $G$ with vertices $v_0,\dots,v_M$ and edges $e_0,\dots,e_L\subseteq C$.
  Assume $G$ contains a cycle with vertex sequence $v_{i_1},\dots,v_{i_K}$ and edges $e_{i_1},\dots,v_{i_K}$. Then
  \begin{align*}
    \partial E = \bigcup_k e_{i_k}\subseteq C
  \end{align*}
  is the boundary of a domain $E$ for which $f=y_c$ on $\partial E$.By the maximum principle $f=0$ on $E$ and thus
  $f=0$ on $\overline{\Omega}$, a contradiction to the non-degeneracy. Hence $G$ is acyclic and the number of 
  intersections of $C$ with $\partial \Omega$ is at least 4 and thus $\partial\Omega$ is divided into 4 components.
  Now choose 4 neighbouring components as depicted in figure [TODO: insert figure]. Let $A\subseteq\Omega$ be the domain bounded
  by $\omega_1$ and $C$ as in the figure. The maximum principle yields that $\omega_1$ contains a local maximum or minimum of $f$.
  Analogously $\omega_2,\dots,\omega_4$ also contain local extrema. Since the $\partial\omega_i$ cannot be extremal points on $\partial\Omega$
  we can assume without loss of generality (by switching $f$ for $-f$) that $\omega_1$ and $\omega_3$ contain local maxima and $\omega_2 $ and $\omega_4$ local
  minima. By Hopf's lemma we thus have
  \begin{align*}
    \Sigma^-\cap\omega_2\neq\emptyset\neq\Sigma^-\cap\omega_4
  \end{align*}
  and 
  \begin{align*}
    \Sigma^+\cap\omega_1\neq\emptyset\neq\Sigma^+\cap\omega_3
  \end{align*}
  From this the claim follows.
\end{proof}

\subsection*{A proof involving invariant manifolds}
\begin{proof}
  Let $x_0,\dots,x_N$ denote the critical points of $f$. Let $\lambda_i\colon\brk*{a_i,b_i}\to\overline{\Omega}$ for $i\in\brk[c]{1,2}$ parametrise the unstable manifolds of the
  critical point $x_0$ and
  $\lambda_i\colon\brk*{a_i,b_i}\to\overline{\Omega}$ for $i\in\brk[c]{1,2}$ be chosen to parametrise the stable manifolds of $x_0$.
  As in the previous proof we can assume the interval on which the $\lambda_i$ are defined to be maximal. We thus have for
  \begin{align*}
    \lambda_i^-=\lim_{t\to a_i}\lambda(t) \\
    \lambda_i^+=\lim_{t\to b_i}\lambda(t)
  \end{align*}
  that $\lambda_i^\pm\in\brk*[c]{x_0,\dots,x_N,\partial\Omega}$ since the $x_j$ are the sole points on $\Omega\cap\overline{C}$
  at which $Df\perp=0$. Thus all invariant manifolds of all critical points form an directed graph $G$ with vertices $v_1,\dots,v_M$ and 
  edges $e_1,\dots,e_L\subseteq\overline{\Omega}$. Here the direction of the edge is determined by whether $f$ increases or decreases
  along the edge. Our graph is in fact acyclic directed.
  % The vertices corresponding to the critical points are partially ordered by the edges and thus there exists a minimal element 
  % Now each vertex corresponding to a critical point has 2 incoming and 2 outgoing arcs.
  TODO: continue proof
\end{proof}  

\newpage

\section*{The case $n=3$}

\subsection*{The case of a single critical point}

\newpage

\subsection*{The case of dimensions $n=4$}
Define the harmonic function 
\begin{align*}
  f\colon B_1\subseteq\R^4&\to \R \\
  x &\mapsto x_1^2+x_2^2-x_3^2-x_4^2\,.
\end{align*}
This has the origin as a stagnation point. We now claim that the sets $\Sigma^+$ and $\Sigma^-$ are both simply connected, i.e.\
we have a tube in $\R^4$ with throughflow and a stagnation point.

\begin{proof}
To prove this claim we observe that the boundary $\partial B_1$ can be parametrised by the coordinates $\bx = (x_2,x_3,x_4)$
for which we have $\abs{\bx}\leq 1$. By the condition
\begin{align}
  \sum_i x_i^2 = 1\label{eq:n4:ball}
\end{align}
on the boundary $\partial B_1$ we have that $x_1$ is then uniquely determined up to sign. Thus we have have defined parametrisations
\begin{align}
  \begin{aligned}\Sigma^\pm\colon B_1\subseteq\R^3&\to\R \\
  \bx\mapsto x, \pm x_1\geq0
  \end{aligned}\label{eq:n4:parametrisation}
\end{align}
We now calculate the derivative of $f$
\begin{align*}
  Df = 2\vect{x_1 & x_2 & -x_3 & -x_4}^\top
\end{align*}
and the normal to $\partial B_1$
\begin{align*}
  n = \vect{x_1 & \cdots & x_4}^\top\,.
\end{align*}
Thus we have $x\in\Sigma^\pm$ iff
\begin{align*}
  0<\pm Df\cdot n = \pm 2\brk*{x_1^2+x_2^2-x_3^2-x_4^2}
\end{align*}
Using the condition \eqref{eq:n4:ball} we obtain the equivalent condition
\begin{align*}
  0<\pm 1-2\brk*{x_3^2+x_4^2}
\end{align*}
Define the cylinder
\begin{align*}
  C = \brk[c]*{\bx\in \R^3\colon x_3^2+x_4^2<1/2} = \R\times B_{1/\sqrt{2}}
\end{align*}
If we return to our parametrisation \eqref{eq:n4:parametrisation} we see that we have $\bx\in B_1\cap C$ iff
$\Sigma^\pm(x)\in \Sigma^+$. Analogously  we have that $\bx\in B_1\setminus C$ iff
$\Sigma^\pm(x)\in \Sigma^-$. Taking into account that $x_1=0$ is equivalent to $\bx\in \partial B_1\subseteq \R^2$
the claim follows from a picture.

(TODO: Elaborate here with some argument with homeormophisms)
\end{proof}

\newpage

% TODO: potentially add Irwin, smooth dynamical systems
\section*{Bibliography}
\nocite{*}
%Main source
%\printbibliography[heading=none, keyword={main}]
%\noindent Other sources
\printbibliography[heading=none]


\end{document}
