

\input{../Latex_Templates/Preamble_Report}

%%%%% TITLE PAGE

%\subject{, VT23}
\title{ Some title \\[1ex]
	  \large Master Thesis}
%\subtitle{}
\author{Theo Koppenhöfer}
\date{Lund \\[1ex] \today}

\addbibresource{bibliography.bib}

\usepackage{pythonhighlight}
\usepackage{pgfplots}
\graphicspath{{../Plots/}}
\graphicspath{{../Figures/}}

\pgfplotsset{
	compat=newest,
    every axis/.append style={
        axis y line=left,
        axis x line=bottom,
        scale only axis,
%    	max space between ticks=25pt,
        width=0.7\textwidth,
        scaled ticks=true,
        axis line style={thick,-,>=latex, shorten >=-.4cm},
    		x tick label style={
		    /pgf/number format/precision=3
		}
    },
    every axis plot/.append style={thick},
    tick style={black, thick}
}

\newcommand{\bx}{\bar{x}}

%%%%% The content starts here %%%%%%%%%%%%%


\begin{document}

\maketitle

Some amazing introduction
 
\newpage

\section*{Some general remarks}

remarks:
\begin{itemize}
  \item only finitely many critical points possible
  \item state Hopf's lemma
\end{itemize}

\subsection*{General definitions}

Define:
\begin{itemize}
  \item emergent, entrant boundary
  \item admissable function, non-degeneracy
\end{itemize}

\subsection*{On assuming non-degeneracy}

\newpage

\section*{Harmonic functions, $n=2$}

\begin{claim}
  Let $\Omega$ be homoemorph to $B_1\subseteq\R^2$. Let further $f\colon\overline{\Omega}\to\R$ be harmonic and admissable 
  as in Morse with critical point $x_1\in\Omega$. Then $\Sigma^-\subseteq\partial\Omega$ consists of at least 2 components.
\end{claim}


\subsection*{A proof involving level-sets}

\begin{proof}[Sketch of Proof.]
  Let $y_c=f(x_1)$ and $x_1,\dots,x_M$ be all the critical points such that $f(x_\cdot)=y_c$.
  We claim that the set of level curves
  \begin{align*}
    C=\brk[c]*{f=y_c}\subseteq\overline{\Omega}
  \end{align*}
  divides the boundary $\partial\Omega$ into 4 components. To show this let $\gamma_i\colon \brk*{a_i,b_i}\to C$ for $i\in\brk[c]{1,\dots,4}$ parametrise the curves in
  $C$ intersecting at $x_1$. These can be constructed with the initial value problem
  \begin{align*} 
    \gamma'&=\brk*{Df}^\perp\big\vert_\gamma \\
    \gamma(0) &= \gamma_0
  \end{align*}
  where $\gamma_0\in C$ is chosen sufficiently near $x_1$. Without loss of generality the intervals on which the $\gamma_i$ are
  defined are maximal. We thus have for
  \begin{align*}
    \gamma_i^-=\lim_{t\to a_i}\gamma(t) \\
    \gamma_i^+=\lim_{t\to b_i}\gamma(t)
  \end{align*}
  that $\gamma_i^\pm\in\brk[c]*{x_1,\dots,x_M,\partial\Omega}$ since the $x_j$ are the sole points on $\Omega\cap\overline{C}$
  at which $Df^\perp=0$. This argument can be applied to all of the 
  $x_1,\dots,x_M$. We therefore have a situation similar to the one depicted in figure \ref{fi:n2:levelSets:overview}.
  
  \begin{figure}
    \centering
    \def\svgwidth{0.6\textwidth}
    \input{../Figures/n2_levelSets_overview.pdf_tex}
    \caption{An overview.}
    \label{fi:n2:levelSets:overview}
  \end{figure}

  One sees that $C$ can thus be represented by a multigraph $G$ with vertices $v_1,\dots,v_K$ and edges $e_1,\dots,e_L\subseteq C$.
  Assume $G$ contains a cycle with vertex sequence $v_{i_1},\dots,v_{i_J}$ and edges $e_{i_1},\dots,e_{i_J}$. Then
  \begin{align*}
    \partial E = \bigcup_j e_{i_j}\subseteq C
  \end{align*}
  is the boundary of a domain $E$ for which $f=y_c$ on $\partial E$. By the maximum principle $f=0$ on $E$ and thus
  $f=0$ on $\overline{\Omega}$, a contradiction to the non-degeneracy. Hence $G$ is acyclic and the number of 
  intersections of $C$ with $\partial \Omega$ is at least 4 and thus $\partial\Omega$ is divided into 4 components.
  
  Now choose 4 neighbouring components as depicted in figure [TODO: insert figure]. 
  Let $A\subseteq\Omega$ be the domain bounded
  by $\omega_1$ and $C$ as in the figure. The maximum principle yields that $\omega_1$ contains a local maximum or minimum of $f$ since
  $f$ is constant on the other boundaries of $A$.
  By the same argument $\omega_2,\dots,\omega_4$ also contain local extrema. 
  TODO: use argument with $\nabla f$ here to show that extrema can be assumed to be alternating.
  Since the $\partial\omega_i$ cannot be extremal points on $\partial\Omega$
  we can assume without loss of generality (by switching $f$ for $-f$) that $\omega_1$ and $\omega_3$ contain local maxima and $\omega_2 $ and $\omega_4$ local
  minima. By Hopf's lemma we thus have
  \begin{align*}
    \Sigma^-\cap\omega_2\neq\emptyset\neq\Sigma^-\cap\omega_4
  \end{align*}
  and 
  \begin{align*}
    \Sigma^+\cap\omega_1\neq\emptyset\neq\Sigma^+\cap\omega_3
  \end{align*}
  From this the claim follows.
\end{proof}

\subsection*{A proof involving invariant manifolds}

Using invariant manifolds we obtain the following proof.

\begin{proof}[Sketch of Proof.]
  Let $x_1,\dots,x_M$ denote the critical points of $f$. Let $\lambda_i\colon\brk*{a_i,b_i}\to\overline{\Omega}$ for $i\in\brk[c]{1,2}$ parametrise the unstable manifolds of the
  critical point $x_1$ and
  $\lambda_i\colon\brk*{a_i,b_i}\to\overline{\Omega}$ for $i\in\brk[c]{3,4}$ be chosen to parametrise the stable manifolds of $x_1$.
  As in the previous proof we can assume the interval on which the $\lambda_i$ are defined to be maximal. We thus have for
  \begin{align*}
    \lambda_i^-=\lim_{t\to a_i}\lambda(t) \\
    \lambda_i^+=\lim_{t\to b_i}\lambda(t)
  \end{align*}
  that $\lambda_i^\pm\in\brk[c]*{x_1,\dots,x_M,\partial\Omega}$ since the $x_j$ are the sole points on $\overline{\Omega}$
  at which $Df=0$. Thus all invariant manifolds of all critical points form a directed multigraph $G$ with vertices $v_1,\dots,v_K$ and 
  edges $e_1,\dots,e_L\subseteq\overline{\Omega}$. Here the direction of the edge is determined by whether $f$ increases or decreases
  along the edge. Here we exclude edges along the boundary $\partial\Omega$.
  By construction graph is acyclic directed. We claim that the underlying undirected graph is in fact a forest. Thus it remains to
  be shown that the underlying undirected graph is acyclic.
  Assume not, i.e.\ we have a undirected cycle with vertices $x_{i_1},\dots,x_{i_J}$ and edges $e_{i_1},\dots,e_{i_J}$.
  Since the set of cycles forms a partial ordering with respect to the property `contains another cycle' we can choose this cycle such
  that it contains no other distinct cycles, i.e.\ it is a minimal cycle.
  We note that each vertex has 2 incoming and 2 outgoing arcs which lie opposite to each other.
  We also note that the edges cannot cross. 
  We can thus describe the trail $x_{i_1},\dots,x_{i_J}$ by a set of directives of the type
  \begin{align*}
    (d_1,\dots,d_K) \in \brk[c]*{\text{l},\text{r},\text{s}}^J\,.
  \end{align*}
  Here l, r and s stand for `left', `right' and `straight' respectively.
  The underlying idea is that we follow a particular trail and orient all vertices as in figure \ref{fi:n2:invariantMF:directivesDefinition}.
  
  \begin{figure}
    \centering
    \begin{minipage}{0.4\textwidth}
      \def\svgwidth{0.7\textwidth}
      \input{../Figures/n2_invariantMF_directivesDefinition.pdf_tex}
      \caption{Explanation of the directives `l',`r' and `r'.}
      \label{fi:n2:invariantMF:directivesDefinition}
    \end{minipage}
    \hfill
    \begin{minipage}{0.4\textwidth}
      \def\svgwidth{1\textwidth}
      \input{../Figures/n2_invariantMF_directivesExample.pdf_tex}
      \caption{An example for a cycle.}
      \label{fi:n2:invariantMF:directivesExample}
    \end{minipage}
  \end{figure}

  An example of the trail `srsr' is given in figure \ref{fi:n2:invariantMF:directivesExample}.
  We now note that cycles of the type r,\dots,r or l,\dots,l
  cannot occur as we otherwise would have a directed cycle. Thus there exists a vertex where the chosen direction is s. 
  Without loss of generality this vertex is $x_{i_1}$.
  Since we can swap $f$ with $-f$ we can assume without loss of generality that the cycle lies to right of $x_{i_1}$.
  Now we look at the directives r,\dots,r
  Since all vertices lie within the cycle we must at some step reach a vertex on the cycle. But then this cycle is a new distinct cycle
  contained in the outer cycle, a contradiction.
  Hence every case considered leads to a contradiction and it follows that the underlying undirected multigraph of $G$
  is acyclic.
  
  We now call a leaf positive if it lies on the emergent boundary and
  negative if it lies on the entrant boundary. The case that a leaf is neither positive or negative cannot occur (elaborate).
  We now pick a tree $\widetilde{G}$ out of $G$ and note that
  there are at least 4 boundary vertices to this tree.
  By construction we see that each `neighbouring' (elaborate) leaf of this tree
  has opposite signage and the claim follows.
\end{proof} 

% We claim that the above proof can be generalised to show the following
% \begin{conjecture}
%   Let $\Omega\subseteq\R^2$ be a regular domain with Betti numbers $R_0=1$ and $R_1$.
%   Let further $f\colon\overline{\Omega}\to\R$ be harmonic and admissable 
%   as in Morse with $M$ critical points. Assume further that 
%   $\overline{\Sigma^-}\subseteq\partial\Omega$ on a given connected component 
%   of the boundary $\partial\Omega$
%   consists of at most 1 connected component. Then we have
%   \begin{align*}
%     2R_1\geq M\,.
%   \end{align*}
% \end{conjecture}

\newpage

\section*{Harmonic vector fields, $n=2$}

\subsection*{No inflow or outflow}

\begin{itemize}
  \item define inflow, outflow
  \item define harmonic vector field
  \item define minimal cycle 
  \item show that $u$ is the gradient of a harmonic function if the domain is simply connected.
  \item discuss what it means for a critical point to be non-degenerate.
\end{itemize}

\begin{proposition}
  Let $\Omega$ be a regular domain with Betti numbers $R_0=1$, and $R_1$ and let $u\colon\Omega\to\R^2$ be 
  a harmonic vector field without inflow or outflow on $\partial\Omega$. Let $M$ denote the number of critical points of $u$.
  Then we have $$M+1\leq R_1$$.
\end{proposition}
\begin{proof}[Sketch of proof.]
  As in previous proofs the critical manifolds form a directed multigraph. Since no critical manifold can intersect with the boundary each vertex
  of the graph has degree 4 and we thus have $2M$ edges. Now we obtain with Euler's polyhedron formula
  for a planar graph with multiple components
  \begin{align*}
    \text{\# interiour minimal cycles}
    &= \text{\#faces}-1 \\
    &= 1+\text{\#components}-\text{\#vertices}+\text{\#edges} -1 \\
    &\geq 1+1-M+2M -1 = M+1
  \end{align*}
  Now note that each interiour minimal cycle must contain a hole of the domain since else we could restrict $u$ to a simply
  connected region containing this cycle. Then $u$ would correspond to the gradient of 
  a harmonic function and we would obtain a contradiction as in the previous proof.
  Hence the number of minimal cycles is a lower bound on the number of holes $R_1$ of the domain.
\end{proof}

We now claim that the inequality in the previous proposition cannot be improved by giving an example of a harmonic vector field
for which $M=R_1-1$.
For this consider the field defined by
\begin{align*}
  u\colon\R^2\setminus\brk*{\brk[c]*{\vect{1\\0},\dots,\vect{m\\0}}}&\to\R^2 \\
  x&\mapsto\sum_{m=1}^M\nabla^\perp\Phi_2\brk*{x-\vect{m \\ 0}}
\end{align*}
where
\begin{align*}
  \Phi_2=\log(\abs{\cdot})
\end{align*}
is the fundamental solution of $\Delta$ on $\R^2$. This is a harmonic vector field since
\begin{align*}
  \curl \nabla^\perp\Phi_2(\cdot-y)= -\Delta\Phi_2(\cdot-y)=0
\end{align*}
and by the spherical symmetry of $\Phi_2$
\begin{align*}
  \diver \nabla^\perp\Phi_2(\cdot-y) = \brk*{\partial_1^2-\partial_2^2}\Phi_2(\cdot-y)=0\,.
\end{align*}

\section*{Harmonic functions, $n=3$}

\subsection*{The case of a single critical point}

\newpage

\subsection*{Harmonic functions, $n=4$} 
Define the harmonic function 
\begin{align*}
  f\colon B_1\subseteq\R^4&\to \R \\
  x &\mapsto x_1^2+x_2^2-x_3^2-x_4^2\,.
\end{align*}
This has a stagnation point at the origin. We now claim that the sets $\Sigma^+$ and $\Sigma^-$ are both simply connected, i.e.\
we have a tube in $\R^4$ with throughflow and a stagnation point.

\begin{proof}
To prove this claim we observe that the boundary $\partial B_1$ can be parametrised by the coordinates $\bx = (x_2,x_3,x_4)$
for which we have $\abs{\bx}\leq 1$. By the condition
\begin{align}
  \sum_i x_i^2 = 1\label{eq:n4:ball}
\end{align}
on the boundary $\partial B_1$ we have that $x_1$ is then uniquely determined up to sign. Thus we have have defined parametrisations
\begin{align}
  \begin{aligned}\phi_\pm\colon B_1\subseteq\R^3&\to\R \\
  \bx&\mapsto x\text{ such that } \pm x_1\geq0
  \end{aligned}\label{eq:n4:parametrisation}
\end{align}
with inverses $\psi_\pm = \brk*{\phi_\pm}^{-1}$.
We now calculate the gradient of $f$
\begin{align*}
  \nabla f = 2\vect{x_1 & x_2 & -x_3 & -x_4}^\top
\end{align*}
and the normal to $\partial B_1$
\begin{align*}
  n = \vect{x_1 & \cdots & x_4}^\top\,.
\end{align*}
Thus we have $x\in\Sigma^\pm$ iff
\begin{align*}
  0<\pm \nabla f\cdot n = \pm 2\brk*{x_1^2+x_2^2-x_3^2-x_4^2}
\end{align*}
Using the condition \eqref{eq:n4:ball} we obtain the equivalent condition
\begin{align*}
  0<\pm 1-2\brk*{x_3^2+x_4^2}
\end{align*}
Define the cylinder
\begin{align*}
  C = \brk[c]*{\bx\in \R^3\colon x_3^2+x_4^2<1/2} = \R\times B_{1/\sqrt{2}}
\end{align*}
If we return to our parametrisation \eqref{eq:n4:parametrisation} we see that we have $\bx\in B_1\cap C$ iff
$\phi_\pm(x)\in \Sigma^+$ and hence 
\begin{align*}
  B_1\cap C=\psi_\pm\brk*{\Sigma^+}\,.
\end{align*}
Analogously  we have 
\begin{align*}
  B_1\setminus C=\psi_\pm\brk*{\Sigma^-}\,.
\end{align*}
The claim then follows from the fact that $\phi$ is a homeomorphism onto its image and $x_1=0$ is 
equivalent to $\bx\in \partial B_1\subseteq \R^2$. The situation is depicted in figure \ref{fi:n4_sigma}.

\begin{figure}
  \centering
  \def\svgwidth{0.7\textwidth}
  \input{../Figures/n4_sigma.pdf_tex}
  \caption{Visualisation of the situation.}
  \label{fi:n4_sigma}
\end{figure}
\end{proof}

\newpage

% TODO: potentially add Irwin, smooth dynamical systems
\section*{Bibliography}
\nocite{*}
%Main source
%\printbibliography[heading=none, keyword={main}]
%\noindent Other sources
\printbibliography[heading=none]


\end{document}